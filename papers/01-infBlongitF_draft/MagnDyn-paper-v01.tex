\documentclass[12pt, reqno]{amsart}
\usepackage{geometry} % see geometry.pdf on how to lay out the page. There's lots.
\geometry{letter} %{a4paper} % or letter or a5paper or ... etc
% \geometry{landscape} % rotated page geometry

% See the ``Article customise'' template for come common customisations

\title{Magnetized Dynamic Friction Force for Times Short Compared to Plasma Period}
\author{Ilya V. Pogorelov and David L. Bruhwiler}
\date{} % delete this line to display the current date

%%% BEGIN DOCUMENT
\begin{document}

\maketitle
\tableofcontents

\section{Introduction and Background}
%\subsection{} 


-- EIC (eRHIC) 

-- magnetized electron cooling 

-- dynamical friction 

-- theoretical and semi-empirical models: Derbenev-Skrinsky, Parkhomchuk;  Meshkov's asymptotic formulas 

-- high-energy, short-interaction-time regime 

\section{Parameter Regime, Reduced Model for the Magnetized Dynamics, and Simulation Setup}

\subsection{Reduced 1D Model in the Limit of Strong Magnetic Field } 
~\\
%\newine
-- parameter regime; a test case with parameters from Fedotov et al. 

-- overall approach as set forth by Derbenev: separating the total force on the ion into the bulk-fields, friction, and statistical-fluctuations contributions 

-- removing the bulk forces by subtracting the forces from electrons on un-perturbed trajectories from the forces due to electrons on the ion-perturbed trajectories 

-- strong magnetic field: canonical perturbation theory predicts orbit gyro-centers staying on cylinders of constant radius (impact parameter) 

-- in the limit of infinitely strong magnetic field, an effective 1D, weakly-nonlinear potential for the longitudinal motion of the electrons 

\subsection{Simulation Setup for Computing Ensemble-Average Friction Force Values} 
~\\
-- track pairs of electron macroparticles with pairwise-identical initial conditions, one on the ion-perturbed trajectory, the other following the trajectory not  perturbed by the ion 

-- calculate time-averaged difference between the forces on the ion from the electron macroparticles that comprise such pairs 

-- initial conditions uniformly sample the e-beam phase space density 

-- first, perform the computation for the (longitudinally) cold electron gas 

-- force for the warm gas calculated by convolution 

Once the friction force is known for the ion interacting with the cold electrons, the dynamic friction force acting on the ion from a gas of warm electrons can be computed by convolution, 
\begin{equation}
F^{(w)}_{\parallel}(V_i) = \int_{-\infty}^{\infty}  F^{(c)}_{\parallel}(V_i -v_e) f(v_e) dv_e ,
\end{equation}
with superscripts $w$ and $c$ indicating the force computed for the ion interacting with the warm and cold electron gas, respectively, and the (arbitrary) distribution density of electrons in the longitudinal velocity space denoted by $f(v_e)$. 



\section{Simulation Results for the Longitudinal Dynamic Friction Force}
%\subsection{} 

-- gold ions and protons 

-- no divergence seen for either small or large impact parameters 

-- linear in V for small V;  interaction time-independent,  $\sim Z^2 V^{-2} $ tail for large $V$ 

-- dependence on $T_{int}$ 

-- $Z^{4/3}$ dependence of the peak force magnitude on the ion charge  

-- positrons (i.e., electron cooling of antiprotons) 


\section{A parametrized-Fit Model for the Longitudinal Dynamic Friction Force}
%\subsection{} 

-- a two-parameter model agrees with computed results to $10-15\%$ 

-- parameters $A$ and $\sigma$ found via fit and analysis of dimensions and scaling behavior 

-- captures the scaling in $n_e, Z, T_{int}$, and $V$ of the friction force in the entire range of ion velocities from zero to very high 

-- physical system has 3 independent parameters; a suitable 3-parameter model, if found, could provide a better parametrized fit  

There are 3 independent parameters in the physical model considered above: the ion charge number $Z$, the local electron number density $n_e$, and the interaction time in the cooler $T_{int}$ in the beam frame.  By construction of our numerical procedure, the computed expectation value of the dynamic friction force depends linearly on $n_e$.  We studied the dependence of the friction force on the ion velocity and the model parameters by computing $F_{\parallel}(V_{\parallel})$ for the gold ion ($Z = 79$) and the protons ($Z = 1$); for protons, the calculation was done for several values of $T_{int}$.  By dimensional analysis and examining the scaling of the peak force as well as $F_{\parallel}(V_{\parallel})$ in the large and small $V_{\parallel}$ limits, we found the following two-parameter model to ...


\begin{equation}
F_{\parallel}(v) = - \frac{A v}{(\sigma^2 + v^2)^{3/2}} ,
\end{equation}
where the two model parameters are given by 
\begin{equation*}
A = 2 \pi Z^2 n_e m_e (r_e c^2)^2 
\end{equation*}
and
\begin{equation*}
\sigma \approx (\pi Z r_e c^2 / T_{int})^{1/3} .
\end{equation*}
The dependence of the peak force on the ion charge in this parametric fit model agrees with the empirically found $|F_{\parallel,max}| \sim Z^{4/3}$ scaling. In addition, the parametric model predicts the dependence of the peak friction force on the interaction time, 
\begin{equation}
|F_{\parallel,max}| \sim Z^{4/3} T_{int}^{2/3}.
\end{equation}
As can be seen in Fig. .... (for the gold ion), the parametric fit model has   v, $Z^2 / V^2$, underestimates peak by .... \%



\section{Comparisons with Analytic and Semi-Analytic Models}
%\subsection{} 

-- exact agreement with the Derbenev-Skrinsky model's asymptotic for large $V$ and infinite $B$ (cold electrons) 

-- for cold electrons, our model agrees with Parkhomchuk for low $V$, but predicts significantly lower dynamic friction force values at intermediate (near peak-force) and high ion velocity values 

 -- for warm electrons,  for both protons and gold ions, our calculations result in generally lower friction force than what is predicted by the Parkhomchuk model, but the situation is [conclusions are] not clear-cut at lower ion velocity 
 
\section{Analytic Calculation in the Limit of Strong Field and Very Large and Very Small Ion Velocity}



\section{Summary and Discussion}
%\subsection{} 

We're confused as much as ever, but on a higher level and about more complicated things 

\end{document}


























